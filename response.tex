\documentclass{article}

\usepackage{url}

\title{Review Reply to JMLR Machine Learning Open Source Software Submission (14-324):\\
BayesPy: Variational Bayesian Inference in Python}

\author{Jaakko Luttinen\\
  \url{jaakko.luttinen@aalto.fi}\\
    Aalto University, Finland}

\begin{document}

\maketitle

\section{Responses to the Reviewers of Submission 14-324 (v1)}

I want to express my gratitude to the action editor and all three reviewers for
their in-depth evaluation and constructive comments.  I have taken into account
the reviews and revised the paper.  In addition, I have added some new features
to BayesPy and released a new version (0.3.1).  The four major changes in this
revised submission can be listed as:
\begin{itemize}

\item BayesPy now supports several advanced VB techniques: stochastic
  variational inference, collapsed inference, Riemannian conjugate gradient
  learning, pattern searches and deterministic annealing.

\item The extremely simple example in the previous submission has been replaced
  by a Gaussian mixture model example.

\item A detailed comparison to Infer.NET has been added.

\item The reported installation issues have been fixed.

\end{itemize}
More detailed responses for each reviewer are given below.


% \subsection{Decision (meta-review)}

% The blah blah

% \begin{itemize}

% \end{itemize}


\subsection{Reviewer 1}

\begin{itemize}

\item \underline{supported models}
  \\
  The revised version makes it more clear that BayesPy supports conjugate
  exponential family models and does not currently support non-conjugate models.
  An example of a non-supported model is given in section 4.  Examples of
  supported popular models are listed in section 2.  Section 3 also provides an
  example implementation of a Gaussian mixture model, which is much more
  interesting model than the very simple model in the previous version of the
  paper.  Because of the page limit, more examples are provided in the online
  documentation (bayespy.org).  Section 4 also discusses Gaussian mixture and
  PCA models, which are given implementations in the supplementary material.
  The results in the section also provide some rough idea about computational
  costs and how large models can be constructed without using stochastic
  variational inference.

\item \underline{speed comparison}
  \\
  The revised version contains a more detailed comparison with Infer.NET.  The
  other relevant active packages are basically MCMC packages.  Because it is not
  obvious how to weigh different aspects of MCMC and VB methods in general, a
  comparison with the MCMC packages is out of the scope of this paper.  However,
  because Infer.NET seems to offer practically the same functionality, section 4
  provides a detailed comparison between BayesPy and Infer.NET, including a
  speed comparison.

\item \underline{scheduling}
  \\
  The updates (and all computations) are performed sequentially, that is,
  variables are updated in turns.  This is now mentioned in Section 3.  Parallel
  computing is not supported.

\item \underline{syntax}
  \\
  The API in the online documentation provides details about the syntax for
  nodes and methods.  For instance, more details about the syntax of the gating
  node has been added in the documentation.  However, it is typically better to
  use Mixture node directly to achieve mixture distributions, as shown in the
  example in section 3.

\end{itemize}

\subsection{Reviewer 2}

\begin{itemize}

\item \underline{installation}
  \\
  The installation instructions in the online documentation (bayespy.org) have
  been improved.  Because BayesPy itself is written purely in Python and thus
  does not need compiling, the only challenge is in installing the requirements
  (a recent NumPy stack).  In order to make the installation of the NumPy stack
  much easier, the instructions mention the possibility to use pre-built
  versions of the NumPy stack (e.g., Anaconda or Enthought).  This is quite
  close to having a stand-alone executable as suggested.

  The reviewer was not able to use the package because the execution halted.
  This was caused by an unfortunate regression bug in matplotlib 1.4.0 and a
  workaround for it in BayesPy.  These issues have now been fixed.  I want to
  thank the reviewer for reporting the bug.

  The error in the second installation seems to be caused by a server error in
  PyPI website (or altenatively some issue in matplotlib), according to my
  understanding.  I hope the reviewer does not experience this issue anymore.

\item \underline{speed comparison}
  \\
  The revised version contains a more detailed comparison with Infer.NET.  The
  other relevant active packages are basically MCMC packages.  It is not obvious
  how to weigh different aspects of MCMC and VB methods in general, thus a
  comparison to the MCMC packages is out of the scope of this paper.  However,
  because Infer.NET seems to offer practically the same functionality, section 4
  provides a detailed comparison between BayesPy and Infer.NET, including a
  speed comparison.

\item \underline{user community}
  \\
  It is difficult to give any accurate estimates of the size of the user
  community.  However, based on given stars, the number of forks and reported
  issues on Github, there seems to be interested users.  In addition, the number
  of downloads in pypi.org and mloss.org give some idea about the number of
  users.

\item \underline{section headings}
  \\
  Section 2 is now labelled ``Features''.

\end{itemize}

\subsection{Reviewer 3}

Reviewer 3 did not point out any modifications the author should make.


\end{document}


%%% Local Variables: 
%%% mode: latex
%%% TeX-PDF-mode: t
%%% TeX-master: t
%%% End: 
