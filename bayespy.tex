\documentclass[twoside,11pt]{article}

% Any additional packages needed should be included after jmlr2e.
% Note that jmlr2e.sty includes epsfig, amssymb, natbib and graphicx,
% and defines many common macros, such as 'proof' and 'example'.
%
% It also sets the bibliographystyle to plainnat; for more information on
% natbib citation styles, see the natbib documentation, a copy of which
% is archived at http://www.jmlr.org/format/natbib.pdf

\usepackage{jmlr2e}
\usepackage{doi}

% Definitions of handy macros can go here

% Heading arguments are {volume}{year}{pages}{submitted}{published}{author-full-names}

\jmlrheading{?}{2014}{?-?}{?/??}{?/??}{Jaakko Luttinen}

% Short headings should be running head and authors last names

\ShortHeadings{BayesPy: Variational Bayesian Inference in Python}{Luttinen}
\firstpageno{1}

\begin{document}

\title{BayesPy: Variational Bayesian Inference in Python}

\author{\name Jaakko Luttinen \email jaakko.luttinen@aalto.fi \\
       \addr Department of Information and Computer Science\\
       Aalto University, Finland}

\editor{?}

\maketitle

\begin{abstract}%   <- trailing '%' for backward compatibility of .sty file
  This paper describes BayesPy.
\end{abstract}

\begin{keywords}
  Bayesian Networks
\end{keywords}

\section{Introduction}

Bayesian framework provides theoretically solid and consistent way to construct
models and perform inference.  In practice, however, the inference is usually
analytically intractable and is therefore based on approximation methods such as
variational Bayes (VB) \cite{?}, expectation propagation (EP) \cite{?} and
Markov chain Monte Carlo (MCMC) \cite{?}.  Deriving and implementing the
formulas for the approximation method is often straightforward but tedious, time
consuming and error prone.  Thus, automatization of this process ..


BayesPy is a Python package providing tools for constructing Bayesian models and
performing variational Bayesian inference easily and efficiently.  It is based
on variational message passing (VMP) framework which defines a simple and local
message passing protocol \cite{?}.  This enables implementation of small general
modules that can be used as building blocks for large complex models.  BayesPy
offers a comprehensive collection of built-in blocks that can be used to build a
wide range of models.  It is written for Python 3 and released under the GNU
General Public License v3.0 (GPLv3).


Several other projects have similar goals for making Bayesian inference easier
and faster for the users.  VB inference is available in Bayes Blocks \cite{?},
VIBES \cite{?} and Infer.NET \cite{?}.  Bayes Blocks is an open-source
C++/Python package but it is limited to scalar Gaussian nodes, a few nonlinear
functions and fully factorial posterior approximation.  VIBES is an old software
package for Java, released under the revised BSD license, but it is no longer
actively developed.  VIBES has been replaced by Infer.NET, which supports


\section{Design and Features}

- numpy, scipy, matplotlib, h5py

- documentation

- installation

- repository

- unit tests

- parameter expansion

- missing values

- examples: PCA, LSSM, HMM, mixture models

Blaa blaa \cite{Luttinen:2013}.


\section{Example}

- example

\section{Conclusion and Future Work}

- future work
-- ep, gibbs
-- ml, map




% Acknowledgements should go at the end, before appendices and references

\acks{?}


\vskip 0.2in
\bibliography{bibliography/bibliography.bib}

\end{document}

%%% Local Variables: 
%%% mode: latex
%%% TeX-PDF-mode: t
%%% TeX-master: t
%%% End: 
